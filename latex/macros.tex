%% This is my base style for my latex documents.

%% Color

\usepackage{color}
\usepackage{xcolor}

\definecolor{red}{rgb}{1,0,0}
\definecolor{green}{rgb}{0,1,0}
\definecolor{darkgreen}{rgb}{0,0.5,0}
\definecolor{blue}{rgb}{0,0,1}
\definecolor{violet}{rgb}{0.4,0,0.8}

\newcommand{\red}[1]{\textcolor{red}{#1}}
\newcommand{\green}[1]{\textcolor{green}{#1}}
\newcommand{\darkgreen}[1]{\textcolor{darkgreen}{#1}}
\newcommand{\blue}[1]{\textcolor{blue}{#1}}
\newcommand{\violet}[1]{\textcolor{violet}{#1}}


\hypersetup{
	colorlinks,
	linkcolor={red!50!black},
	citecolor={blue!50!black},
	urlcolor={blue!80!black}
}

%% Basic algebra

\newcommand{\N}{\mathbb{N}}

\newcommand{\Z}{\mathbb{Z}}
\newcommand{\Zn}[1]{\mathbb{Z}/#1\mathbb{Z}}
\newcommand{\F}{\mathbb{F}}
\newcommand{\Q}{\mathbb{Q}}

\newcommand{\R}{\mathbb{R}}
\newcommand{\Rn}{\R^n}

\newcommand{\C}{\mathbb{C}}

\newcommand{\K}{\mathbb{K}}

\newcommand{\pgcd}[2]{\text{pgcd}(#1,#2)}

\newcommand{\goatp}{\mathfrak{p}}
\newcommand{\goatm}{\mathfrak{m}}

\newcommand{\car}{\text{car}}

\newcommand{\degExt}[2]{[ #1 : #2 ]}

\newcommand{\id}{\text{id}}


%% Polynomials

\newcommand{\QX}{\mathbb{Q}[X]}
\newcommand{\RX}{\mathbb{R}[X]}
\newcommand{\CX}{\mathbb{C}[X]}

\newcommand{\AX}{A[X]}

\newcommand{\Pt}{\tilde{P}}
\newcommand{\Qt}{\tilde{Q}}


%% Vector spaces

\newcommand{\sprod}[2]{\left\langle #1, #2 \right\rangle}

\newcommand{\LE}{\mathcal{L}(E)}

\newcommand{\spec}{\text{Spec}}

\newcommand{\abs}[1]{\left|#1\right|}
\newcommand{\norm}[1]{\abs{\abs{#1}}}
\newcommand{\normop}[1]{\abs{\abs{\abs{#1}}}}

\newcommand{\dist}[2]{\text{ d}\left(#1, #2\right)}
\newcommand{\vect}[1]{\text{Vect}\left( #1 \right)}

\newcommand{\Mn}[2]{\mathcal{M}_{#1}(#2)}

%% Probability notation

\newcommand{\Pro}{\mathscr{P}}
\newcommand{\E}{\mathbb{E}}

\newcommand{\1}{\mathds{1}}


%% Functions

\newcommand{\restr}[2]{#1_{\restriction #2}}
\newcommand{\im}[1]{\text{Im}\left(#1\right)}

%% Logic

\newcommand{\set}[1]{\left\{#1\right\}}
\newcommand{\setdef}[2]{\set{#1 \mid #2}}
\newcommand{\parts}[1]{\mathcal{P}\left(#1\right)}


\newcommand{\contradict}{\lightning}

%% Abbreviations

\newcommand{\ie}{\textit{i.e.} }
\newcommand{\eg}{\textit{e.g.} }


%% Environments


%% Title

\usepackage{ifthen}

\newcommand{\titlepageY}[3]{%
	\fancypagestyle{toc}{%
		\fancyhf{}%
		\fancyhead[L]{\rightmark}%
		\fancyhead[R]{\thepage}%
	}

	\pagestyle{toc}

	\begin{titlepage}
		\newcommand{\HRule}{\rule{\linewidth}{0.5mm}}
		\center

		\HRule\\[0.4cm]

		\textsc{\Large #1}\\[0.5cm]
		\ifthenelse{ \equal{#3}{} }
		{ }
		{\textsc{\large #3}\\[0.5cm] }

		\HRule\\[1.5cm]


		{\large\textit{Auteur}}\\
		#2


		\vfill\vfill\vfill

		{\large\today}

		\vfill

	\end{titlepage}
}



